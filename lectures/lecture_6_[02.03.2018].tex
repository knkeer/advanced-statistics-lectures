% !TeX root = ../main.tex

\section{Лекция 6}
На этой лекции разбирались задачи из первого домашнего задания. Здесь мы оставим только формулировки.
\begin{problem}
    Пусть $\vec{X} = (X_{1}, \ldots, X_{n})$ "--- выборка из равномерного распределения на отрезке $[\,0, \theta]$. Проверьте на несмещенность, состоятельность, сильную состоятельность и асимптотическую нормальность следующие оценки параметра $\theta$: $2\overline{\vec{X}}$, $\overline{\vec{X}} + X_{(n)}/2$, $(n + 1)X_{(1)}$, $X_{(1)} + X_{(n)}$, $\frac{n + 1}{n}X_{(n)}$.
\end{problem}

\begin{problem}
    Пусть $\vec{X} = (X_{1}, \ldots, X_{n})$ "--- выборка из гамма-распределения с параметрами $(\alpha, \lambda)$. Предложите асимптотически нормальную оценку $\alpha > 0$ и вычислите ее асимптотическую дисперсию, если 
    \begin{enumerate}[label=(\alph*)]
        \item $\lambda$ известно;
        \item $\lambda$ тоже неизвестно.
    \end{enumerate}
\end{problem}

\begin{problem}
    Пусть $\vec{X} = (X_{1}, \ldots, X_{n})$ "--- выборка из равномерного распределения на отрезке $[0, \theta]$. Сравните следующие оценки параметра $\theta$ в равномерном подходе с квадратичной функцией потерь:
    \begin{enumerate}
        \item $\hat{\theta}_{1}(\vec{X}) = 2\overline{\vec{X}}$;
        \item $\hat{\theta}_{2}(\vec{X}) = X_{(1)} + X_{(n)}$;
        \item $\hat{\theta}_{3}(\vec{X}) = \frac{n + 1}{n}X_{(n)}$.
    \end{enumerate}
\end{problem}

\begin{problem}
    Пусть $\theta^{*}_{1}(\vec{X})$ и $\theta^{*}_{2}(\vec{X})$ — две ``почти наилучшие'' оценки параметра $\theta$ в среднеквадратичном походе (т.е. каждая из них не хуже любой другой оценки), имеющие одинаковые математические ожидания. Докажите, что тогда для любого $\theta$ они совпадают почти наверное, т.е. $\theta^{*}_{1}(\vec{X}) = \theta^{*}_{2}(\vec{X})$ $\Pr_{\theta}$-п.н.
\end{problem}

\begin{problem}
    $\vec{X} = (X_{1}, \ldots, X_{n})$ "--- выборка из распределения $\mathcal{N}(\theta, 1)$, $\theta > 0$. Сравните в равномерном подходе относительно квадратичной функции потерь оценки $\overline{\vec{X}}$ и $\max(0, \overline{\vec{X}})$.
\end{problem}

\begin{problem}
    Пусть $\vec{X} = (X_{1}, \ldots, X_{n})$ "--- выборка из гамма-распределения с плотностью
    \[
        p_{\theta}(x) 
        = \frac{2^{\theta}}{\Gamma(\theta)}x^{\theta - 1}e^{-2x}[x \geq 0].
    \]
    где $\theta > 0$ "--- неизвестный параметр. Для каких функций $\tau(\theta)$ существует эффективная оценка? Найдите информацию Фишера $i(\theta)$ одного элемента выборки.
\end{problem}

\begin{problem}
    Пусть $\vec{X} = (X_{1}, \ldots, X_{n})$ "--- выборка из нормального распределения с параметрами $(\mu, \sigma^{2})$. Найдите эффективную оценку
    \begin{enumerate}[label=(\alph*)]
        \item параметра $\mu$, если $\sigma$ известно;
        \item параметра $\sigma^{2}$, если $\mu$ известно.
    \end{enumerate}
    Вычислите информацию Фишера одного наблюдения в обоих случаях. Найдите информационную матрицу в случае, когда оба параметра $\mu$ и $\sigma^{2}$ неизвестны.
\end{problem}

\begin{problem}
    Пусть $\vec{X}$ "--- наблюдение из ``регулярного'' семейства $\set{\Pr_{\vec{\theta}} \mid \vec{\theta} \in \Theta}$, $\Theta \subseteq \mathbb{R}^{k}$, $k > 1$. Докажите, что если $\hat{\vec{\theta}}(\vec{X})$ "--- эффективная оценка $\vec{\theta}$, то она является оценкой максимального правдоподобия для $\vec{\theta}$.
\end{problem}

\begin{problem}
    Пусть $\vec{X} = (X_{1}, \ldots, X_{n})$ "--- выборка из распределения с плотностью
    \[
        p_{\vec{\theta}}(x) = \frac{1}{\alpha}\exp\left\{-\frac{x - \beta}{\alpha}\right\}[x \geq \beta],
    \]
    где $\vec{\theta} = (\alpha, \beta)$, $\alpha > 0$ "--- двумерный параметр. Найдите для $\vec{\theta}$ оценку максимального правдоподобия. Докажите, что полученная для $\alpha$ оценка $\hat{\alpha}_{n}(\vec{X})$ является асимптотически нормальной, и найдите ее асимптотическую дисперсию.
\end{problem}