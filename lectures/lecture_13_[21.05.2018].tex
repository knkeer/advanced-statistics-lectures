%!TEX root = ../main.tex

\section{Лекция 13}
Вернёмся к параметрическому критерию хи-квадрат. Ранее мы сказали, что при выполнении определённых условий будет выполнен аналог теоремы Пирсона. Сформулируем их:
\begin{enumerate}[label=\alph*)]
	\item $\Theta \subseteq \mathbb{R}^{s}$, $s < m$ "--- открытое множество.
	\item Для любого значения параметра все вероятности отделены от нуля: $p_{i}(\theta) \geq c^{2} > 0$.
	\item Будем считать, что $\pdv{p_{i}(\theta)}{\theta_{j}}$ и $\pdv[2]{p_{i}(\theta)}{\theta_{j}}{\theta_{k}}$ непрерывны на всём $\Theta$.
	\item Матрица $D = \|\pdv{p_{i}(\theta)}{\theta_{j}}\|_{i, j = 1}^{m, s}$ имеет ранг $s$ для любого $\theta \in \Theta$.
\end{enumerate}
Теперь можно сформулировать теорему про параметрический критерий хи-квадрат.
\begin{theorem}
	Пусть выполнены условия модели. Введём следующую систему уравнений:
	\begin{equation}
		\sum_{i = 1}^{m} \frac{\mu_{i}}{p_{i}(\theta)}\pdv{p_{i}(\theta)}{\theta_{j}} = 0, \quad j = 1, \ldots, s
	\end{equation}
	Если верна гипотеза $\mathrm{H}_{0}$, то с вероятностью, стремящейся к 1, данная система имеет единственное решение $\hat{\theta}(\vec{X})$ такое, что $\hat{\theta}(\vec{X})$ сходится по вероятности к истинному значению параметра $\theta$ и
	\begin{equation}
		\hat{\chi}^{2}_{n}(\vec{X}) = \sum_{i = 1}^{m} \frac{(\mu_{j} - np_{j}(\hat{\theta}(\vec{X})))^{2}}{np_{j}(\hat{\theta}(\vec{X}))} \xrightarrow[n \to \infty]{d} \chi^{2}_{m - 1 - s}.
	\end{equation}
\end{theorem}	