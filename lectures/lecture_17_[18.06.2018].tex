%!TEX root = ../main.tex

\section{Лекция 17}
\subsection{Ранговые методы}
Вернёмся к критериям однородности, но несколько изменим формулировку. Пусть $\vec{X} = (X_{1}, \ldots, X_{m})$ и $\vec{Y} = (Y_{1}, \ldots, Y_{n})$ "--- независимые выборки с некоторыми неизвестными непрерывными распределениями. Можно ли сказать, что $X_{i}$ равно $Y_{j}$ по распределению для всех $i = 1, \ldots, m$, $j = 1, \ldots, n$?

\begin{definition}
	Соберём выборки $\vec{X}$ и $\vec{Y}$ в вектор $\vec{Z}$. \emph{Рангом} случайной величины $Y_{j}$ называется
	\begin{equation}
		R(Y_{j}) = \sum_{i = 1}^{n + m} \mathbf{1}_{Z_{i} \leq Y_{j}}.
	\end{equation}
\end{definition}
\begin{definition}
	\emph{Статистикой ранговых сумм Вилкоксона} называется 
	\begin{equation}
		W_{m, n}(\vec{X}, \vec{Y}) = \sum_{j = 1}^{n} R(Y_{j}).
	\end{equation}
\end{definition}
\begin{statement}
	Если верна гипотеза однородности, то статистика ранговых сумм Вилкоксона не зависит от распределения выборок.
\end{statement}
\begin{proof}
	Пусть $F(x)$ "--- функция распределения элементов выборок. Так как она непрерывна, то $\mathbf{1}_{X_{i} \leq Y_{j}} = \mathbf{1}_{F(X_{i}) \leq F(Y_{j})}$. Но $F(X_{1}) \sim \mathrm{U}[0, 1]$.
\end{proof}